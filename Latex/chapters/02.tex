\chapter{Related literature}
% \begin{itemize}
%     \item  problem analysis
%     \item state of the art, problem statement
%     \item  literature research (all sources in the thesis have to be referenced)
%     \item description of existing solutions (also scientific ones, if the problem is scientifically researched), algorithms,  location of the thesis in the scientific domain
% \end{itemize}
The purpose of this thesis is to create and validate an integrated deep learning pipeline for compression of HSIs, with particular focus on Earth observation satellites.
Therefore, before developing the models, it is crucial to understand what are the particularities of working with HSIs, the challenges they present, and the existing methods for their compression.
The literature review provides an overview of metrics used in image compression evaluation, general concepts of image compression, and a survey of existing HSI compression models. % wlasnie metryki trzeba będzie przenieść 
Reviewed literature includes scientific articles and conference papers, and is described in more detail in the following sections. % to zdanie jest imo niepotrzebne 
These publications were selected for their relevance to the topic of this thesis and were screened to ensure only current research that has not been superseded was used.
These papers served as the first step in the development of model implementations.

\section{Processing challenges of hyperspectral images}
Hyperspectral image acquisition enables the capture of detailed spectral information, but it introduces challenges related to data volume and transmission.
A single HSI can be comprised of hundreds of spectral bands for each spatial pixel, leading to substantial file sizes.
This poses significant challenges for storage, processing, and transmission, especially in resource-constrained environments such as satellites and remote sensing platforms. % to i pierwsze zdanie to maslo maslane 
Furthermore, as the technology advances, the number of spectral bands and the spatial resolution of hyperspectral images continue to increase, exacerbating the problem.
From the signal processing perspective, HSIs exhibit high correlation across both spatial and spectral dimensions~\cite{HighSpatioSpectralCorrelation}.
This correlation can be exploited for compression, but it also means that traditional compression techniques may not be optimal.

\section{Approaches to image compression}
Image compression is a technique used to reduce the size of images for storage and transmission purposes.
The main objective of image compression is to minimize the amount of data required to represent an image.
The most important aspects of image compression are the compression ratio (CR) and the quality of the reconstructed image.
The compression ratio is defined as the ratio of the original image size to the compressed image size.
A higher CR means a greater reduction in size~\cite{lossless_comprehensive_review2025}.
Compression can be broadly categorized into two types: lossy and lossless~\cite{lossless_comprehensive_review2025}.
Lossless compression techniques allow for the exact reconstruction of the original image from the compressed data.
These techniques are essential in applications where any loss of information is unacceptable~\cite{CCSDS123Review2021}.
However, lossless compression typically achieves lower compression ratios compared to lossy methods or takes more computational resources.
Lossy compression techniques, on the other hand, allow for some loss of information in exchange for higher compression ratios~\cite{CCSDS123Review2021}.
These techniques are commonly used in applications where some loss of quality is acceptable, such as web images and videos.
Lossy compression methods allow for higher compression ratios than lossless methods, but they may introduce artifacts and degrade image quality.
The choice between lossy and lossless compression depends on the specific requirements of the application, including the acceptable level of quality loss and the desired compression ratio.\par

\section{Hyperspectral image compression}
Hyperspectral image compression has been an active area of research, with various approaches proposed over the years.
These methods can be loosely categorized into traditional compression techniques and deep learning-based methods.
Traditional compression techniques include, but are not limited to, methods such as statistical coding, wavelet transforms, and predictive coding ~\cite{HSICompressionReview2015}.
These methods often rely on hand-crafted features and may not fully exploit the complex correlations present in hyperspectral data. % to sie musze zastanowić jak przerobić 
In recent years, deep learning-based methods have gained prominence due to their ability to learn complex representations from data~\cite{lossless_comprehensive_review2025}. % Chociaz idk czy inne reviewy nie bylyby lepsze
Autoencoders ~\cite{HySpecNet11k}, convolutional neural networks (CNNs)~\cite{CNNForHSI}, and generative adversarial networks (GANs)~\cite{GANForHSI} have been employed for hyperspectral image compression.
These methods can adaptively learn to compress hyperspectral images based on the data distribution, potentially leading to better performance compared to traditional methods.
Several studies have demonstrated the effectiveness of deep learning-based compression methods for hyperspectral images~\cite{HySpecNet11k, CNNForHSI}.
These methods are known to exploit the high spatial and spectral correlation present in HSIs, leading to improved compression ratios and reconstruction quality~\cite{lossless_comprehensive_review2025}.


\subsection{Algorithmic approaches}
HSI compression algorithms can be distinguished based on the techniques they approach.
One of the approaches is prediction-based, which relies on sequentially estimating pixel values based on previously encoded values.
Due to its simplicity, this approach is often used in lossless compression algorithms~\cite{CCSDS123Review2021}.
The CCSDS 123.0-B-2 standard is an example of a predictive coding algorithm, which is the standard for lossless and near-lossless compression in space applications due to their low computing requirements~\cite{CCSDS123Review2021} \par
A major recent breakthrough is a deep learning model called LineRWKV~\cite{valsesia2024linerwkv}.
This model uses a predictive approach but replaces the fixed formula with a neural network.
It became the first deep learning (DL) model proven to beat CCSDS 123.0-B-2 standard by achieving better compression while being efficient enough to run on a satellite~\cite{valsesia2024linerwkv}.\\
Another approach is transformation-based compression.
Instead of looking at pixels individually, this approach looks at the image as a whole and applies a mathematical transformation to repackage the entire image's information into a new, more compact form.\par
One of the most prominent examples of such transformation-based methods in deep learning are autoencoders.
Autoencoders are neural networks trained to squeeze an image into a smaller ``bottleneck'' representation.
This bottleneck is called latent space, and it contains the most important features of the original image in a compressed form.
The process of squeezing is called encoding, and the reverse process of reconstructing the image to its original form is called decoding~\cite{barman2022autoencoders}.
While autoencoders can obtain high compression ratios, due to the small size of the latent space representation, the information in it is heavily rounded off, which makes the process lossy.

\section{Semantic segmentation of hyperspectral images}
Semantic segmentation is a computer vision task that involves classifying each pixel in an image into predefined categories based on image content.
In the context of HSIs, semantic segmentation is a particularly powerful tool due to the rich spectral information available for each pixel.
This, in turn, allows for more accurate and detailed classification compared to traditional RGB images.
HSI semantic segmentation has found applications in various fields, including agriculture~\cite{ environmental_monitoring_hyperspectral2009}, forestry~\cite{ remote_sensing_uav_hyperspectral2017}, geology~\cite{geology2012}, surveillance~\cite{ HSI_security_surveillance2010}, and environmental monitoring~\cite{chemometrics_environmental_monitoring2020}. % papery z 1 rozdzialu
For example, in agriculture applications, semantic segmentation of HSIs is used to identify crop types, monitor nutrients, and detect diseases~\cite{chemometrics_environmental_monitoring2020}.
Thus, semantic segmentation is crucial for further utilization of HSIs in various applications.
However, the high dimensionality and complexity of HSIs create a significant computational challenge for doing semantic segmentation in real-time or resource-constrained environments.
Therefore, it is essential to ensure swift and efficient transmission and storage of HSIs, which can be achieved through different compression techniques.
In order to evaluate the impact of compression on downstream tasks such as semantic segmentation, it is important to include a segmentation model in the evaluation pipeline.
Due to the focus of this thesis being on compression techniques, a simple segmentation model is used for evaluation purposes.
