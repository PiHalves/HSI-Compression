\chapter{Dataset}% Requirements and tools}

\section{Choosing the dataset}
Among the many difficulties that need to be addressed when creating machine learning models, one of the most significant is sthe election of the dataset.
The dataset must be relevant to the problem, of sufficient size, and of high quality.
An unbalanced dataset or a dataset with too many items of one class can lead to biased models with poor generalization capabilities.
A poorly preprocessed dataset with noise can lead to models that behave unpredictably in real-world scenarios. \par
The size of the dataset is another important factor.
Large datasets are generally preferred as they provide the possibility to learn complex patterns, but it is also important to choose a dataset with diverse examples to cover a wide range of scenarios.
Choosing a large dataset comes with increased computational requirements, longer training time, and storage needs.
Selecting a dataset that is too small carries its own risks.
A model trained on an insufficient dataset may not learn crucial patterns and quickly overfit to the training data, resulting in poor performance on unseen data.
Taking these factors into account, the HySpecNet-11k dataset was chosen, which is described in Section \ref{sec:hyspecnet}.
The dataset was accessed in November 2025, and it has been publicly available since its publication in 2023 \cite{HySpecNet11k}.
This, however, changed in 2026, when the dataset was made unavailable due to licensing issues \cite{HySpecNetDatasetUnavailable}.

\section{HySpecNet 11k dataset} \label{sec:hyspecnet}
For training, validation, and testing of models, the HySpecNet-11k dataset was used \cite{HySpecNet11k}.
The dataset contains 11,483 nonoverlapping image patches. Each patch is \(128\times128\) with 224 spectral bands.
All of the images were captured in the real world by the Environmental Mapping and Analysis Program (EnMAP) satellite with a ground sampling distance of 30 meters. %info from the HySpecNet paper
The number of bands is reduced from 224 to 202 by removing bands in the ranges [127-141] and [161-167] due to high water vapor absorption. \par %info from the HySpecNet paper
The dataset is split into two categories, easy and hard.
The easy split was used for training and testing the models, as the primary task is data compression rather than image segmentation or classification. % zobaczyć czy to ok
Each category is divided into training, validation, and testing subsets with a ratio of 70\%, 20\%, and 10\,% respectively.
The dataset is distributed in the form of geotiff files, with each file containing several bands of a single image patch.
To work with the dataset, first, the patches had to be reconstructed from the GeoTIFF files into 3D NumPy arrays of shape (128, 128, 202).
To facilitate this, the code provided by the dataset authors was used~\cite{HySpecNet11k}.

\section{Data visualization}
To better understand the dataset and visualize the hyperspectral images, a custom visualization tool was developed.
The tool allows selecting individual image patches from the dataset and displaying them in single-band and RGB formats.
Figures \ref{fig:hyspecnet-rgb-examples} and \ref{fig:hyspecnet-bands} show examples of visualizations generated by the tool.
\begin{itemize}
    \item Figure \ref{fig:hyspecnet-rgb-examples} present an RGB visualization of three different image patches obtained by connecting bands 43, 28 and 10 at wavelengths 634 nm, 550 nm and 463 nm.
    \item Figure \ref{fig:hyspecnet-bands} present a set of 4 images showing individual bands from a single image patch.
\end{itemize}

% !!! ADD FIGURES !!!
% FIGURE FROM FILE
%
%\begin{figure}
%\centering
%\includegraphics[width=0.5\textwidth]{./graf/politechnika_sl_logo_bw_pion_en.pdf}
%\caption{Caption of a figure is always below the figure.}
%\label{fig:label}
%\end{figure}
%Fig. \ref{fig:label} presents …

%%%%%%%%%%%%%%%%%%%%
%% SUBFIGURES
%
\begin{figure}
    \centering
    \begin{subfigure}{0.3\textwidth}
        \includegraphics[width=\textwidth]{./graf/ENMAP01-____L2A-DT0000004950_20221103T162438Z_001_V010110_20221118T145147Z-Y07860913_X09511078-DATA_rgb_normalized.png}
        % \caption{Upper left figure.}
        \label{fig:hyspecnet-rgb-1}
    \end{subfigure}
    \hfill
    \begin{subfigure}{0.3\textwidth}
        \includegraphics[width=\textwidth]{./graf/ENMAP01-____L2A-DT0000004950_20221103T162438Z_001_V010110_20221118T145147Z-Y09141041_X05670694-DATA_rgb_normalized.png}
        % \caption{Upper right figure.}
        \label{fig:hyspecnet-rgb-2}
    \end{subfigure}
    \hfill
    \begin{subfigure}{0.3\textwidth}
        \includegraphics[width=\textwidth]{./graf/ENMAP01-____L2A-DT0000005022_20221104T152758Z_018_V010110_20221118T174025Z-Y07730900_X07910918-DATA_rgb_normalized.png}
        % \caption{Lower left figure.}
        \label{fig:hyspecnet-rgb-3}
    \end{subfigure}

    \begin{subfigure}{0.3\textwidth}
        \includegraphics[width=\textwidth]{./graf/ENMAP01-____L2A-DT0000005031_20221102T025823Z_013_V010110_20221116T205052Z-Y05410668_X01770304-DATA_rgb_normalized.png}
        % \caption{Upper left figure.}
        \label{fig:hyspecnet-rgb-4}
    \end{subfigure}
    \hfill
    \begin{subfigure}{0.3\textwidth}
        \includegraphics[width=\textwidth]{./graf/ENMAP01-____L2A-DT0000005117_20221107T113507Z_002_V010110_20221119T095812Z-Y05300657_X06730800-DATA_rgb_normalized.png}
        % \caption{Upper right figure.}
        \label{fig:hyspecnet-rgb-5}
    \end{subfigure}
    \hfill
    \begin{subfigure}{0.3\textwidth}
        \includegraphics[width=\textwidth]{./graf/ENMAP01-____L2A-DT0000005122_20221107T162842Z_002_V010110_20221117T145318Z-Y02820409_X08200947-DATA_rgb_normalized.png}
        % \caption{Lower left figure.}
        \label{fig:hyspecnet-rgb-6}
    \end{subfigure}

    \caption{Example images from the dataset colored to RGB using bands 43, 28 and 10 corresponding to wavelengths 634 nm, 550 nm and 463 nm respectively.}
    \label{fig:hyspecnet-rgb-examples}
\end{figure}
% Fig. \ref{fig:subfigures} presents very important information, eg. Fig. \ref{fig:upper-right} is an upper right subfigure.
%%%%%%%%%%%%%%%%%%%%%
%
\begin{figure}
    \centering
    \begin{subfigure}{0.6\textwidth}
        \includegraphics[width=\textwidth]{./graf/ENMAP01-____L2A-DT0000004950_20221103T162438Z_001_V010110_20221118T145147Z-Y07860913_X09511078-DATA_rgb_normalized.png}
        % \caption{Upper left figure.}
        \caption{RGB visualization using bands 43, 28 and 10 corresponding to wavelengths 634 nm, 550 nm and 463 nm respectively.}
        \label{fig:hyspecnet-rgb-7}
    \end{subfigure}

    \begin{subfigure}{0.3\textwidth}
        \includegraphics[width=\textwidth]{./graf/ENMAP01-____L2A-DT0000004950_20221103T162438Z_001_V010110_20221118T145147Z-Y07860913_X09511078-DATA_band_100_normalized.png}
        % \caption{Upper left figure.}
        \caption{Single band visualization of band 100}
        \label{fig:hyspecnet-single-band-1}
    \end{subfigure}
    \hfill
    \begin{subfigure}{0.3\textwidth}
        \includegraphics[width=\textwidth]{./graf/ENMAP01-____L2A-DT0000004950_20221103T162438Z_001_V010110_20221118T145147Z-Y07860913_X09511078-DATA_band_150_normalized.png}
        % \caption{Upper right figure.}
        \caption{Single band visualization of band 150}
        \label{fig:hyspecnet-single-band-2}
    \end{subfigure}
    \hfill
    \begin{subfigure}{0.3\textwidth}
        \includegraphics[width=\textwidth]{./graf/ENMAP01-____L2A-DT0000004950_20221103T162438Z_001_V010110_20221118T145147Z-Y07860913_X09511078-DATA_band_200_normalized.png}
        % \caption{Lower left figure.}
        \caption{Single band visualization of band 200}
        \label{fig:hyspecnet-single-band-3}
    \end{subfigure}
    \caption{Example image patch from the dataset visualized in RGB using bands 43, 28 and 10 \ref{fig:hyspecnet-rgb-7} and three individual bands: band 100 \ref{fig:hyspecnet-single-band-1}, band 150  \ref{fig:hyspecnet-single-band-2} and band 200 \ref{fig:hyspecnet-single-band-3}.}
    \label{fig:hyspecnet-bands}

\end{figure}



