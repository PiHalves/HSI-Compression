\subsubsection*{Thesis title}
\Title

\subsubsection*{Abstract}
    Hyperspectral images (HSIs) provide detailed spectral information across hundreds of bands but generate significantly larger data volumes than traditional RGB images, creating challenges for storage and transmission in resource-constrained environments such as satellites. 
    This thesis explores deep learning-based compression methods for HSIs, implementing and evaluating four models: LineRWKV for lossless compression, and RCAE2D1D, RCAE3D, and RCGDNAE for lossy compression.
    The models were trained and tested on the HySpecNet11k dataset, with performance assessed using metrics such as Bits Per Pixel Per Band (BPPPB), Peak Signal-to-Noise Ratio (PSNR), and Structural Similarity Index Measure (SSIM).
    Another important aspect of this work is the evaluation of the feasibility of these models for downstream tasks, specifically semantic segmentation, which was assessed using a separate model trained on the original, uncompressed data.
    The results indicate that deep learning-based compression techniques can effectively reduce HSI data sizes while maintaining differing quality for both visual assessment and semantic segmentation tasks depending on the model with some models excelling in visual quality and others in segmentation performance.

\subsubsection*{Key words}
Deep Learning, Compression, Hyperspectral Images

\subsubsection*{Tytuł pracy}
\begin{otherlanguage}{polish}
    \TitleAlt
\end{otherlanguage}

\subsubsection*{Streszczenie}
\begin{otherlanguage}{polish}
    Obrazy hiperspektralne (HSI) dostarczają szczegółowych informacji spektralnych w setkach pasm, ale generują znacznie większe wolumeny danych niż tradycyjne obrazy RGB, co stwarza wyzwania związane z przechowywaniem i transmisją w środowiskach o ograniczonych zasobach, takich jak satelity.
    Niniejsza praca bada metody kompresji oparte na uczeniu głębokim dla HSI, implementując i oceniając cztery modele: LineRWKV do kompresji bezstratnej oraz RCAE2D1D, RCAE3D i RCGDNAE do kompresji stratnej.
    Modele zostały wytrenowane i przetestowane na zbiorze danych HySpecNet11k, a wydajność oceniono za pomocą metryk takich jak bity na piksel na pasmo (BPPPB), szczytowy stosunek sygnału do szumu (PSNR) oraz wskaźnik podobieństwa strukturalnego (SSIM).
    Kolejnym ważnym aspektem tej pracy jest ocena wykonalności tych modeli dla zadań następczych, w szczególności segmentacji semantycznej, która została oceniona za pomocą oddzielnego modelu wytrenowanego na oryginalnych, nieskompresowanych danych.
    Wyniki wskazują, że techniki kompresji oparte na uczeniu głębokim mogą skutecznie zmniejszyć rozmiary danych HSI, zachowując różną jakość zarówno do oceny wizualnej, jak i zadań segmentacji semantycznej w zależności od modelu, przy czym niektóre modele wyróżniają się jakością wizualną, a inne wydajnością segmentacji.

\end{otherlanguage}

\subsubsection*{Słowa kluczowe}
\begin{otherlanguage}{polish}
    Uczenie Głębokie, Kompresja, Obrazy Hiperspektralne
\end{otherlanguage}

